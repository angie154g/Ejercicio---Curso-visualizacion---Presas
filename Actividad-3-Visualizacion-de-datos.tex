% Options for packages loaded elsewhere
\PassOptionsToPackage{unicode}{hyperref}
\PassOptionsToPackage{hyphens}{url}
%
\documentclass[
]{article}
\usepackage{amsmath,amssymb}
\usepackage{iftex}
\ifPDFTeX
  \usepackage[T1]{fontenc}
  \usepackage[utf8]{inputenc}
  \usepackage{textcomp} % provide euro and other symbols
\else % if luatex or xetex
  \usepackage{unicode-math} % this also loads fontspec
  \defaultfontfeatures{Scale=MatchLowercase}
  \defaultfontfeatures[\rmfamily]{Ligatures=TeX,Scale=1}
\fi
\usepackage{lmodern}
\ifPDFTeX\else
  % xetex/luatex font selection
\fi
% Use upquote if available, for straight quotes in verbatim environments
\IfFileExists{upquote.sty}{\usepackage{upquote}}{}
\IfFileExists{microtype.sty}{% use microtype if available
  \usepackage[]{microtype}
  \UseMicrotypeSet[protrusion]{basicmath} % disable protrusion for tt fonts
}{}
\makeatletter
\@ifundefined{KOMAClassName}{% if non-KOMA class
  \IfFileExists{parskip.sty}{%
    \usepackage{parskip}
  }{% else
    \setlength{\parindent}{0pt}
    \setlength{\parskip}{6pt plus 2pt minus 1pt}}
}{% if KOMA class
  \KOMAoptions{parskip=half}}
\makeatother
\usepackage{xcolor}
\usepackage[margin=1in]{geometry}
\usepackage{graphicx}
\makeatletter
\newsavebox\pandoc@box
\newcommand*\pandocbounded[1]{% scales image to fit in text height/width
  \sbox\pandoc@box{#1}%
  \Gscale@div\@tempa{\textheight}{\dimexpr\ht\pandoc@box+\dp\pandoc@box\relax}%
  \Gscale@div\@tempb{\linewidth}{\wd\pandoc@box}%
  \ifdim\@tempb\p@<\@tempa\p@\let\@tempa\@tempb\fi% select the smaller of both
  \ifdim\@tempa\p@<\p@\scalebox{\@tempa}{\usebox\pandoc@box}%
  \else\usebox{\pandoc@box}%
  \fi%
}
% Set default figure placement to htbp
\def\fps@figure{htbp}
\makeatother
\setlength{\emergencystretch}{3em} % prevent overfull lines
\providecommand{\tightlist}{%
  \setlength{\itemsep}{0pt}\setlength{\parskip}{0pt}}
\setcounter{secnumdepth}{-\maxdimen} % remove section numbering
\usepackage{bookmark}
\IfFileExists{xurl.sty}{\usepackage{xurl}}{} % add URL line breaks if available
\urlstyle{same}
\hypersetup{
  hidelinks,
  pdfcreator={LaTeX via pandoc}}

\author{}
\date{\vspace{-2.5em}}

\begin{document}

\#--- \#title: ``Actividad 3 Visualizacion de datos'' \#author:
``Angélica Benítez'' \#date: ``2025-05-22'' \#output: word\_document
\#editor\_options: markdown: \# wrap: 72 \#---

\#librerias

library(dplyr) library(tidyr) library(readr) library(ggplot2)

\#1 Verificar si los datos se pueden leer \# Leer un archivo CSV y
\#guardarlo como un data frame marzo\_25 \textless-
read.csv(``Reporte\_Presas\_2025-03-01.csv'') marzo\_24 \textless-
read.csv(``Reporte\_Presas\_2024-03-01.csv'') marzo\_23 \textless-
read.csv(``Reporte\_Presas\_2023-03-01.csv'') marzo\_15 \textless-
read.csv(``Reporte\_Presas\_2015-03-01.csv'') marzo\_14 \textless-
read.csv(``Reporte\_Presas\_2014-03-01.csv'') marzo\_13 \textless-
read.csv(``Reporte\_Presas\_2013-03-01.csv'') marzo\_05 \textless-
read.csv(``Reporte\_Presas\_2005-03-01.csv'') marzo\_04 \textless-
read.csv(``Reporte\_Presas\_2004-03-01.csv'') marzo\_03 \textless-
read.csv(``Reporte\_Presas\_2003-03-01.csv'') marzo\_95 \textless-
read.csv(``Reporte\_Presas\_1995-03-01.csv'') marzo\_94 \textless-
read.csv(``Reporte\_Presas\_1994-03-01.csv'') marzo\_93 \textless-
read.csv(``Reporte\_Presas\_1993-03-01.csv'')

\section{2. Mostrar las primeras filas del data
frame}\label{mostrar-las-primeras-filas-del-data-frame}

head(marzo\_25)

\section{3. tipos de variables
presentes}\label{tipos-de-variables-presentes}

str(marzo\_25)

\#4. tiene datos nulos(NA) any(is.na(marzo\_25)) any(is.na(marzo\_24))
any(is.na(marzo\_23)) any(is.na(marzo\_15)) any(is.na(marzo\_14))
any(is.na(marzo\_13)) any(is.na(marzo\_05)) any(is.na(marzo\_04))
any(is.na(marzo\_03)) any(is.na(marzo\_95)) any(is.na(marzo\_94))
any(is.na(marzo\_93))

\#5. Agrupa los datos por decadas \# 5.1 Agrega una columna a cada
\#dataframe con la fecha en que se obtienen \# los datos, es parte del
\#nombre de archivo marzo\_25 \textless- marzo\_25 \%\textgreater\%
mutate(año = 2025, decada = 2020) marzo\_24 \textless- marzo\_24
\%\textgreater\% mutate(año = 2024, decada = 2020) marzo\_23 \textless-
marzo\_23 \%\textgreater\% mutate(año = 2023, decada = 2020) marzo\_15
\textless- marzo\_15 \%\textgreater\% mutate(año = 2015, decada = 2010)
marzo\_14 \textless- marzo\_14 \%\textgreater\% mutate(año = 2014,
decada = 2010) marzo\_13 \textless- marzo\_13 \%\textgreater\%
mutate(año = 2013, decada = 2010) marzo\_05 \textless- marzo\_05
\%\textgreater\% mutate(año = 2005, decada = 2000) marzo\_04 \textless-
marzo\_04 \%\textgreater\% mutate(año = 2004, decada = 2000) marzo\_03
\textless- marzo\_03 \%\textgreater\% mutate(año = 2003, decada = 2000)
marzo\_95 \textless- marzo\_95 \%\textgreater\% mutate(año = 1995,
decada = 1990) marzo\_94 \textless- marzo\_94 \%\textgreater\%
mutate(año = 1994, decada = 1990) marzo\_93 \textless- marzo\_93
\%\textgreater\% mutate(año = 1993, decada = 1990)

\section{5.2. Unir por decadas}\label{unir-por-decadas}

decada\_2020 \textless- bind\_rows(marzo\_25, marzo\_24, marzo\_23)
decada\_2010 \textless- bind\_rows(marzo\_15, marzo\_14, marzo\_13)
decada\_2000 \textless- bind\_rows(marzo\_05, marzo\_04, marzo\_03)
decada\_1990 \textless- bind\_rows(marzo\_95, marzo\_94, marzo\_93)

\section{6. Selecciona solo las columnas de interes: nombre comun,
entidad,}\label{selecciona-solo-las-columnas-de-interes-nombre-comun-entidad}

\section{almacenaje actual, \% de llenado, año y
decada.}\label{almacenaje-actual-de-llenado-auxf1o-y-decada.}

decada\_2020 \textless- decada\_2020 \%\textgreater\%
select(Nombre.común, Entidad.federativa, Almacenamiento.Actual..hm..,
X..de.llenado.actual, decada, año) decada\_2010 \textless- decada\_2010
\%\textgreater\% select(Nombre.común, Entidad.federativa,
Almacenamiento.Actual..hm.., X..de.llenado.actual,decada, año)
decada\_2000 \textless- decada\_2000 \%\textgreater\%
select(Nombre.común, Entidad.federativa, Almacenamiento.Actual..hm..,
X..de.llenado.actual, decada, año) decada\_1990 \textless- decada\_1990
\%\textgreater\% select(Nombre.común, Entidad.federativa,
Almacenamiento.Actual..hm.., X..de.llenado.actual, decada, año)

\#7. conoce la estructura general de los dato con summary
summary(decada\_2020) summary(decada\_2010) summary(decada\_2000)
summary(decada\_1990)

\section{8. buscar existencia de NA}\label{buscar-existencia-de-na}

any(is.na(decada\_2020)) any(is.na(decada\_2020\$X..de.llenado.actual))
any(is.na(decada\_2010)) any(is.na(decada\_2000))
any(is.na(decada\_1990))

\section{8.1 Remover filas con valores
NA}\label{remover-filas-con-valores-na}

decada\_2020\_sin\_na \textless- decada\_2020 \%\textgreater\%
drop\_na() decada\_2010\_sin\_na \textless- decada\_2010
\%\textgreater\% drop\_na() decada\_2000\_sin\_na \textless-
decada\_2000 \%\textgreater\% drop\_na() decada\_1990\_sin\_na
\textless- decada\_1990 \%\textgreater\% drop\_na()

\section{9. convetir a nuemerico la columna \% de llenado actual y la de
almacenamiento}\label{convetir-a-nuemerico-la-columna-de-llenado-actual-y-la-de-almacenamiento}

decada\_2020\_sin\_na \textless- decada\_2020\_sin\_na \%\textgreater\%
mutate(X..de.llenado.actual = as.numeric(X..de.llenado.actual))
decada\_2010\_sin\_na \textless- decada\_2010\_sin\_na \%\textgreater\%
mutate(X..de.llenado.actual = as.numeric(X..de.llenado.actual))
decada\_2000\_sin\_na \textless- decada\_2000\_sin\_na \%\textgreater\%
mutate(X..de.llenado.actual = as.numeric(X..de.llenado.actual))
decada\_1990\_sin\_na \textless- decada\_1990\_sin\_na \%\textgreater\%
mutate(X..de.llenado.actual = as.numeric(X..de.llenado.actual))

\#9.1 confirmar tipo de variable
str(decada\_2020\_sin\_na\(X..de.llenado.actual)
str(decada_2020_sin_na\)Almacenamiento.Actual..hm..)

\section{9.2 Conoce la esctuctura general de los datos con
summary}\label{conoce-la-esctuctura-general-de-los-datos-con-summary}

summary(decada\_2020\_sin\_na)

\#10. Calcular medidas de tendencia central para el almacenamiento
actual

\#10.1 Promedio mean(decada\_2020\_sin\_na\(Almacenamiento.Actual..hm..)
print ("Promedio almacenamiento el 1 de marzo de la decada 2010")
mean(decada_2010\)Almacenamiento.Actual..hm..) print (``Promedio
almacenamiento el 1 de marzo de la decada 2000'')
mean(decada\_2000\(Almacenamiento.Actual..hm..)
print ("Promedio almacenamiento el 1 de marzo de la decada 1990")
mean(decada_1990\)Almacenamiento.Actual..hm..)

\#10.2 Mediana print (``Mediana de almacenamiento el 1 de marzo de la
decada 2020'')
median(decada\_2020\_sin\_na\(Almacenamiento.Actual..hm..)
print ("mediana de almacenamiento el 1 de marzo de la decada 2010")
median(decada_2010\)Almacenamiento.Actual..hm..) print (``mediana de
almacenamiento el 1 de marzo de la decada 2000'')
median(decada\_2000\(Almacenamiento.Actual..hm..)
print ("Mediana de almacenamiento el 1 de marzo de la decada 1990")
median(decada_1990\)Almacenamiento.Actual..hm..)

\#11 Unir todos los datos en un dataframe df\_presas\_total \textless-
bind\_rows(decada\_2020\_sin\_na, decada\_2010\_sin\_na,
decada\_2000\_sin\_na, decada\_1990\_sin\_na)

\#12. Realizar un grafico facetado por decada de las variables
\#almacenamiento y \% de llenado

\#12.1 Almacenamiento actual ggplot(df\_presas\_total, aes(x =
Entidad.federativa, y = Almacenamiento.Actual..hm..)) +
geom\_boxplot(fill = ``steelblue'') + facet\_wrap(\textasciitilde{}
decada) + theme\_minimal() + theme(axis.text.x = element\_text(angle =
45, hjust = 1)) + labs(title = ``Distribución de llenado por entidad y
década'', x = ``Entidad'', y = ``Almacenamiento en hm'')

\#12.2 \% de llenado actual ggplot(df\_presas\_total, aes(x =
X..de.llenado.actual)) + geom\_histogram(bins = 30, fill =
``steelblue'', color = ``white'') + facet\_wrap(\textasciitilde{}
decada, scales = ``free\_y'') + theme\_minimal() + labs(title =
``Distribución de llenado por década'', x = ``\% de llenado actual'', y
= ``Frecuencia'')

\end{document}
